%% Copyright 2019 2047freehk
%%
%% This work may be distributed and/or modified under the 
%% conditions of the LaTeX Project Public License, either 
%% version 1.3 of this license or (at your option) any later 
%% version. The latest version of this license is in
%%
%%   http://www.latex-project.org/lppl.txt
%%
%% and version 1.3 or later is part of all distributions of 
%% LaTeX version 2005/12/01 or later.
%%
%% This work is modified from the package tikzpeople created by Nils Fleischhacker in 2016
%%
%% This work consists of the files 
%% freehk.tex, 
%% freehk.sty,
%% freehk.shape.protestor.sty, 
%% freehk.shape.hkpolice.sty, 
%%
\documentclass{ltxdoc}
\usepackage{freehk}
\usepackage[OT1]{fontenc}
\usepackage{hyperref}
\usepackage{microtype}
\usepackage{xspace}
\usepackage[most]{tcolorbox}
\usepackage[title]{appendix}
\usetikzlibrary{shapes.callouts}

%\newcommand*{\Appendixautorefname}{Appendix}


\pgfdeclarelayer{background}
\pgfsetlayers{background,main}
\selectcolormodel{cmyk}

\newcommand{\tikzname}{Ti\emph{k}Z\xspace}

\newcommand{\varg}[1]{%
	{\ttfamily\char`\{}#1{\ttfamily\char`\}}}
\newcommand{\DescribeOption}[1]{\leavevmode
	\marginpar{\raggedleft\strut\MacroFont\string #1\ }}

\lstdefinestyle{example}{style=tcblatex,
  classoffset=0,
  texcsstyle=*\color{blue},%
  deletetexcs={begin,end},
  moretexcs={,%
    pgfdeclarehorizontalshading,pgfuseshading,node,
    useasboundingbox,draw}%
  classoffset=1,
  keywordstyle=\color{blue},%
  morekeywords={tikzpicture,shade,fill,draw,path,node,child,line,width,rectangle,minimum width,minimum size},
  classoffset=0}

\tcbset{%
  fillbackground/.style={before lower pre={%
  \tikzset{every picture/.style={execute at end picture={\begin{pgfonlayer}{background}
    \fill[yellow!15]
    ([xshift=-4mm,yshift=-4mm]current bounding box.south west) rectangle
    ([xshift=4mm,yshift=4mm]current bounding box.north east);
    \end{pgfonlayer}}}}}},
  explicitpicture/.style={before lower=\begin{center},after lower=\end{center},fillbackground}}

\newtcblisting{codeexample}[2][]{%
  enhanced,frame hidden,
  top=0pt,bottom=0pt,left=0pt,right=0pt,arc=0pt,boxrule=0pt,
  colback=blue!15,
  listing style=example,
  sidebyside,text and listing,text outside listing,sidebyside gap=2mm,
  lefthand width=#2,tikz lower,fillbackground,
  #1}

\title{\tikz{\node[protestor,shield,umbrella,mask,minimum height=2cm]{}}\\The \textsf{freehk} package}
\author{2047 freehk}
\date{}

\begin{document}
\maketitle

\begin{abstract}
  This package provides people shaped pgf-shapes to be used in \tikzname pictures. This is based on tikzpeople package by Nils Fleischhacker. Hence, it is a package but not a \tikzname library.
\end{abstract}

\section{Introduction}
  This package is to remember freedom fighters in Hong Kong.

\subsection{Installation}
 
The installation is similar to that for tikzpeople. It is adapted to this package as below.


	To install the package just drop the \textsf{freehkesty} file and all the shape files, i.e. \textsf{freehk.shape.\meta{shapename}.sty}, somewhere \LaTeX can find them.
	You might for example put them in your local \textsf{texmf} tree at \url{\textasciitilde/texmf/tex/latex/freehk/}.
	
	Alternatively simply drop all the files into the folder where your main \textsf{.tex} file resides.\footnote{Do not use a subfolder. While you can refer to the package itself in a hackish way using a relative path, \LaTeX wont be able to find the shape files.}



\section{Usage}
	To use the \textsf{freehk} shapes, just write |\usepackage|\varg{freehk} somewhere in the preamble of your document.

\subsection{Available Shapes}

Below is also adapted from tikzpeople documentation.
    
Once the package is loaded, any of the defined shapes can be used as the shape of any \tikzname node just like any other predefined shape.
\begin{codeexample}{3cm}
\node[draw,rectangle,minimum size=1.5cm] at (0,2) {};
\node[protestor,minimum size=1.5cm] at (0,0) {};
\end{codeexample}

However, it should be noted that in many respects the nodes behave quite differently from standard node shapes.

First, the nodes are drawn and filled, whether |draw| and |fill| are specified or not, because -- quite frankly -- if you do not want to draw the node, then the \textsf{freehk} are quite useless to you.

Another area where the behavior might be unexpected to the experienced \tikzname user is concerning the node text.
	
\begin{codeexample}{3cm}
\node[draw,rectangle,minimum size=1.5cm] at (0,2) {A Rectangle};
\node[protestor,minimum size=1.5cm] (B) at (0,0) {A protestor};
\draw[gray,dotted] (B.north west) rectangle (B.south east);
\end{codeexample}

	The text appears below the actual node, not within and the width of the text influence neither the drawn shape nor any of the border anchors.
	The reason for this is that in the intended usecase, the node text -- if present at all -- is simply a label and is not supposed to influence the size or behavior of the node.
%	
%	Another oddity of the tikzpeople shapes is that the border anchors and size of the node only takes into consideration the underlying shape of a person, and nothing of the -- sometimes much larger -- stuff such as hats added by some of the shapes.
%	\begin{codeexample}{3cm}
%\node[person,minimum size=1.5cm] (B) at (0,4) {A Person};
%\draw[gray,dotted] (B.north west) rectangle (B.south east);
%\node[mexican,minimum size=1.5cm]     (M) at (0,0) {A Mexican};
%\draw[gray,dotted] (M.north west) rectangle (M.south east);
%	\end{codeexample}
%	Again, the reason for this is that two shapes specified to have the same size, should behave like two people of same size.
%	If for example \texttt{minimum width} would take the actual width into consideration, then the Mexican would be much smaller than the person.
%	
%	While all of these oddities are useful to me and in my specific usecase, they may cause problem for you in any number of ways.
%	So if you don't think you will be able to cope with them, I suggest you stop reading and look elsewhere for people-shaped \tikzname nodes.
%
\subsection{The Available Shapes}

There are two basic shapes: protestor and hkpolice.

\DescribeMacro{protestor}	The |protestor| is on black bloc and wears a yellow hat.
\begin{codeexample}{3.2cm}
\node[protestor,minimum size=1cm,xshift=-1.2cm]{};
\node[protestor,masked,minimum size=1cm] {};
\end{codeexample}

\DescribeMacro{hkpolice}	The |hkpolice| has a white badge without number.
\begin{codeexample}{3.2cm}
\node[hkpolice,minimum size=1cm,xshift=-1.2cm]{};
\node[hkpolice,masked,minimum size=1cm] {};
\end{codeexample}

\subsection{General Node Options}

Most of the options in \textsf{freehk} can be used to influence the appearance. 

\DescribeOption{masked}	The |masked| option add a mask to face.
\begin{codeexample}{3.2cm}
\node[protestor,masked,minimum size=1cm,xshift=-1.2cm]{};
\node[hkpolice,masked,minimum size=1cm]{};
\end{codeexample}

\DescribeOption{umbrella}	The |umbrella| option add an umbrella.
\begin{codeexample}{3.2cm}
\node[protestor,umbrella,minimum size=1cm,xshift=-1.2cm]{};
\node[hkpolice,umbrella,minimum size=1cm]{};
\end{codeexample}

\DescribeOption{sign}	The |sign| option add a sign.
\begin{codeexample}{3.2cm}
\node[protestor,sign,minimum size=1cm,xshift=-1.2cm]{};
\node[hkpolice,sign,minimum size=1cm]{};
\end{codeexample}
The default signpost is number 5. To change the number, we need to use the option signpost.

\DescribeOption{signpost}	The |signpost| option add a sign with custom text.
\begin{codeexample}{3.2cm}
\node[protestor,sign,signpost={HK},minimum size=1cm,xshift=-1.2cm]{};
\node[hkpolice,sign,signpost={HK},minimum size=1cm]{};
\end{codeexample}

\subsection{General Node Options from \textsf{tikzpeople}}

We can use most of the options in \textsf{tikzpeople} to influence the appearance. 

\DescribeOption{female}	The |female| option changes the hair style from male to female.
\begin{codeexample}{3.2cm}
\node[protestor,minimum size=1cm,xshift=-1.2cm]{};
\node[hkpolice,female,minimum size=1cm]{};
\end{codeexample}

\DescribeOption{good}	The |good| option is supposed to make nodes look extraordinarily good.
		For most tikzpeople this is accomplished by adding a halo.
		\begin{codeexample}{3.2cm}
\node[protestor,good,minimum size=1cm,xshift=-1.2cm]{};
\node[hkpolice,good,minimum size=1cm]{};
		\end{codeexample}

	\DescribeOption{evil}	The |evil| option is supposed to make nodes look more evil than usual.
	This is accomplished by adding horns and a goatee.
	\begin{codeexample}{3.2cm}
\node[protestor,evil,minimum size=1cm,xshift=-1.2cm]{};
\node[hkpolice,evil,minimum size=1cm] {};
	\end{codeexample}
The color of the horns is controlled with the |horns| key.


	\DescribeOption{mirrored}	The \textsf{mirrored} option makes nodes face to the left instead of the right.
		\begin{codeexample}{3.2cm}
\node[protestor,mirrored,minimum size=1cm,xshift=-1.2cm]{};
\node[hkpolice,mirrored,minimum size=1cm]{};
		\end{codeexample}
		
	\DescribeOption{monitor} The \textsf{monitor} option draws a monitor in front of the node.
		\begin{codeexample}{4cm}
\node[protestor,monitor,minimum size=1cm,xshift=-1.5cm]{};
\node[hkpolice,monitor,minimum size=1cm]{};
		\end{codeexample}
	
\DescribeOption{sword} The \textsf{sword} option gives the node a sword.
		\begin{codeexample}{4cm}
\node[protestor,sword,minimum size=1cm,xshift=-1.2cm]{};
\node[hkpolice,sword,minimum size=1cm]{};
		\end{codeexample}
		This could be useful to symbolize that a party is attacking or defending something and combines well with the |shield| option.
		The colors of the sword are controlled using the keys |swordblade|, |swordguard|, |swordpommel|, and |swordgrip|.

	\DescribeOption{shield} The \textsf{shield} option gives the node a shield.
		\begin{codeexample}{4cm}
\node[protestor,shield,minimum size=1cm,xshift=-1.2cm]{};
\node[hkpolice,shield,minimum size=1cm]{};
		\end{codeexample}
		This could be useful to symbolize that a party is defending something and combines well with the |sword| option.
		The colors of the shield are controlled using the keys |shieldmid|, |shieldedge|, and |shieldrivets|.
	
	

	All of these options can be arbitrarily combined. 
	\begin{codeexample}{4cm}
\node[protestor,evil,female,good,mirrored,
monitor, mask, umbrella,sword,minimum size=1.5cm]{};
	\end{codeexample}


\subsection{Color Options}
	As in \textsf{tikzpeople}, almost all the colors can be specified separately.
	
	For example, we can have an elderly protestor:
	\begin{codeexample}{3cm}
\node[protestor, shirt=black, hair=gray, minimum size=1.5cm]{};
	\end{codeexample}

See appendix for further color options.

\subsection{Anchors}

Anchor is the same as \textsf{tikzpeople}. What is worth repeating here is is the |mouth| anchor. In conjunction with \tikzname's |callout| shapes this anchor allows to easily depict talking \textsf{freehk}.
	\begin{codeexample}{4.6cm}
\node[name=a,shape=protestor,sign,minimum size=1cm,xshift=-1.25cm] {};
\node[name=b,shape=hkpolice,minimum size=1cm,mirrored,xshift=1.25cm] {};
\node[ellipse callout, draw,yshift= 1.0cm, callout absolute pointer={(a.mouth)}, font=\tiny] {5 demands!};
\node[ellipse callout, draw, yshift=-.3cm, callout absolute pointer={(b.mouth)}, font=\tiny] {What?};
	\end{codeexample}

\clearpage
\begin{appendices}
\section{Color Options}
\freehkcolors{protestor}
\vfill\freehkcolors{hkpolice}
	\end{appendices}
\end{document}

